\documentclass[paper=a4, fontsize=11pt]{scrartcl} % A4 paper and 11pt font size

\usepackage[T1]{fontenc} % Use 8-bit encoding that has 256 glyphs
\usepackage{fourier} % Use the Adobe Utopia font for the document - comment this line to return to the LaTeX default
\usepackage[english]{babel} % English language/hyphenation
\usepackage{amsmath,amsfonts,amsthm} % Math packages

\usepackage{lipsum} % Used for inserting dummy 'Lorem ipsum' text into the template

\usepackage{sectsty} % Allows customizing section commands
\usepackage{pgfplots}
\allsectionsfont{ \normalfont\scshape} % Make the default font and small caps

\usepackage{fancyhdr} % Custom headers and footers
\pagestyle{fancyplain} % Makes all pages in the document conform to the custom headers and footers
\fancyhead{} % No page header - if you want one, create it in the same way as the footers below
\fancyfoot[L]{} % Empty left footer
\fancyfoot[C]{} % Empty center footer
\fancyfoot[R]{\thepage} % Page numbering for right footer
\renewcommand{\headrulewidth}{0pt} % Remove header underlines
\renewcommand{\footrulewidth}{0pt} % Remove footer underlines
\setlength{\headheight}{13.6pt} % Customize the height of the header

\setlength\parindent{0pt} % Removes all indentation from paragraphs - comment this line for an assignment with lots of text

%----------------------------------------------------------------------------------------
%	TITLE SECTION
%----------------------------------------------------------------------------------------

\newcommand{\horrule}[1]{\rule{\linewidth}{#1}} % Create horizontal rule command with 1 argument of height

\title{	
\normalfont \normalsize 
\textsc{CS6220} \\ [25pt] % Your university, school and/or department name(s)
\horrule{0.5pt} \\[0.4cm] % Thin top horizontal rule
\huge Self-Test \\ % The assignment title
\horrule{2pt} \\[0.5cm] % Thick bottom horizontal rule
}

\author{} % Your name

\date{\normalsize2017-09-12} % Today's date or a custom date

\begin{document}

\maketitle % Print the title

%----------------------------------------------------------------------------------------
%	PROBLEM 1
%----------------------------------------------------------------------------------------

\section{Bayes' Rule}

\begin{align*} 
P(rain) = 73/365 = 0.2
\Rightarrow
P(not rain) = 0.8
\end{align*}
\begin{align*} 
p(predictRain|rain) = 0.7
\end{align*}
\begin{align*}
p(predictRain|not rain) = 0.3
\end{align*}

\[
\begin{split} 
P(rain|predictRain)
  &= \dfrac{P(rain)P(predictRain|rain)}{P(predictRain)}
  \\
  &= \dfrac{P(predictRain|rain)P(rain)}{P(predictRain|rain)P(rain) + P(predictRain|not\space rain)P(not\space rain)}
  \\
  &=  \dfrac{0.7*0.2}{0.7*0.2 + 0.3*0.8}
  \\
   &=  0.37
\end{split}
\]

%----------------------------------------------------------------------------------------
%	PROBLEM 2
%----------------------------------------------------------------------------------------

\section{Probability Distributions}

\begin{align*}
A = 
\begin{bmatrix}
A_{11} & A_{21} \\
A_{21} & A_{22}
\end{bmatrix}
\end{align*}
\begin{equation*}\label{SplitFunc}
p(x) = \left\{
	\begin{array}{rcl}
       \int_{0}^{x} 4t \mathrm{d}t & \mbox{for} & 0\leq x \leq1/2 \\
       \int_{0}^{x} (-4t+4) \mathrm{d}t & \mbox{for} & 1/2\leq x \leq1
  	\end{array}\right.
\end{equation*}
\begin{equation*}\label{SplitFunc}
p(x) = \left\{
	\begin{array}{rcl}
       2x^{2}+C_1 & \mbox{for} & 0\leq x \leq1/2 \\
      -2x^{2} + 4x+C_2 & \mbox{for} & 1/2\leq x \leq1
  	\end{array}\right.
\end{equation*}

\begin{align*}
\text{Assume the entire mass is} \in [0, 1] \\
\end{align*}
\begin{align*}
P(0) = 0 \\
P(1) = 1 \\
\end{align*}
\begin{align*}
\text{Plug back into the equation and we get} \\
\end{align*}
\begin{align*}
C_1 = 0 \\
C_2 = -1
\end{align*}


%----------------------------------------------------------------------------------------
%	PROBLEM 3
%----------------------------------------------------------------------------------------

\section{Discrete Expectation}

    \[
        \begin{split}
            E(x)  &= \frac{1}{2}*1 + \frac{1}{10}*2 + \frac{1}{10}*3 + \frac{1}{10}*4 + \frac{1}{10}*5 + \frac{1}{10}*6
            \\
            &= 2.5
        \end{split}
    \]

%----------------------------------------------------------------------------------------
%	PROBLEM 4
%----------------------------------------------------------------------------------------

\section{Expectation Properties}
    \[
        \begin{split}
            Var[x]  &= E[(x-\mu)^2]
\\&= E[(x-E(x))^2]
\\&= E[x^2 - 2xE(x)+ (E(x))^2]
\\&= E[x^2] - 2E(x)E(x)+ (E(x))^2
\\&= E[x^2] - (E(x))^2
        \end{split}
    \]

%----------------------------------------------------------------------------------------
%	PROBLEM 5
%----------------------------------------------------------------------------------------

\section{Matrices/Linear Equations}
a.
    \[
        \begin{split}
\begin{bmatrix}
2 & 1 & 1\\ 
4 & 0 & 2\\ 
1 & 2 & 0
\end{bmatrix} x
=
 \begin{bmatrix}
3\\ 
10\\ 
-2
\end{bmatrix}
        \end{split}
    \]
b.
    \[
        \begin{split}
\begin{bmatrix}
4 & 2 & 2\\ 
4 & 0 & 2\\ 
2 & 2 & 0
\end{bmatrix} x
=
 \begin{bmatrix}
6\\ 
10\\ 
-2
\end{bmatrix}
\Rightarrow 
\begin{bmatrix}
0 & 2 & 0\\ 
4 & 0 & 2\\ 
4 & 4 & 0
\end{bmatrix} x
=
 \begin{bmatrix}
-4\\ 
10\\ 
-4
\end{bmatrix}
\Rightarrow\\
\begin{bmatrix}
0 & 2 & 0\\ 
4 & 0 & 2\\ 
4 & 0 & 0
\end{bmatrix} x
=
 \begin{bmatrix}
-4\\ 
10\\ 
4
\end{bmatrix}
\Rightarrow 
\begin{bmatrix}
0 & 2 & 0\\ 
0 & 0 & 2\\ 
4 & 0 & 0
\end{bmatrix} x
=
 \begin{bmatrix}
-4\\ 
6\\ 
4
\end{bmatrix}\\
s =
 \begin{bmatrix}
1\\ 
-2\\ 
3
\end{bmatrix}
        \end{split}
    \]
    
 c. 
     \[
        \begin{split}
	 \begin{bmatrix}
3\\ 
10\\ 
-2
\end{bmatrix} = 
 \begin{bmatrix}
2\\ 
4\\ 
2
\end{bmatrix} -2
 \begin{bmatrix}
1\\ 
0\\ 
2
\end{bmatrix} + 3
 \begin{bmatrix}
1\\ 
2\\ 
0
\end{bmatrix}
        \end{split}
    \]

%----------------------------------------------------------------------------------------
%	PROBLEM 6
%----------------------------------------------------------------------------------------


\section{Matrices}
a.
    \[
        \begin{split}
            det(A)  &= 1*4*4 + 2*3*1 + 3*1*3 - 1*4*3 + 3*3*1 - 4*1*2
            \\
            &= 2
        \end{split}
    \]

b.
    \[
\begin{bmatrix}
0 & 2 & 0\\ 
1 & 4 & 3\\ 
0 & 1 & -1
\end{bmatrix} x
\Rightarrow
\begin{bmatrix}
0 & 2 & 0\\ 
1 & 4 & 3\\ 
0 & 0 & 2
\end{bmatrix}
    \]

As the matrix have no rows of all 0, therefore the matrix is invertible\\

c.
As the matrix is invertible, the rank equals the number of rows which is 3

%----------------------------------------------------------------------------------------
%	PROBLEM 7
%----------------------------------------------------------------------------------------

\section{Matrices}
    \[
\left | A - \lambda I \right | = \begin{vmatrix}
3-\lambda & 6\\ 
1 & 4-\lambda
\end{vmatrix}
    \]
    \[
for \space \lambda = 1:
\left | A - \lambda I \right | = \begin{vmatrix}
2 & 6\\ 
1 & 3
\end{vmatrix}
\begin{vmatrix}
v1\\
v2
\end{vmatrix}
=
\begin{vmatrix}
0\\
0
\end{vmatrix}
    \]
        
So we don't even need to plug in 6, the answer is b
    \[
\begin{vmatrix}
-3\\
1
\end{vmatrix}
    \]

%----------------------------------------------------------------------------------------
%	PROBLEM 8
%----------------------------------------------------------------------------------------

\section{Matrices}

p = 0
\[
        \begin{split}
		\left \| x \right \|_0 = 4 
        \end{split}
        \begin{align*}
	        \text{(number of non-zero elements)}
        \end{align*}
\]

p = 1
\[
        \begin{split}
\left \| x \right \|_1 =  \sum_{i = 1}^{n} \left | x_i \right |
&= \left | 2 \right | + \left | 1 \right | + \left | -4 \right | + \left | -2 \right |
\\
&= 9
        \end{split}
\]

p = 2
\[
        \begin{split}
\left \| x \right \|_2 =  (\sum_{i = 1}^{n} \left | x_i \right |)^{\!1/2}
&= \sqrt{\left | 2 \right |^2 + \left | 1 \right |^2 + \left | -4 \right |^2 + \left | -2 \right |^2}
\\
&= 5
        \end{split}
\]

    \[
    p = \infty
        \begin{split}
\left \| x \right \|_\infty =  \lim_{p\to\infty}(\sum_{i = 1}^{p} \left | x_i \right |)^{\!1/p}
&= \max_{i}|x_i|_\infty
\\
&= 4
        \end{split}
    \]

\end{document}